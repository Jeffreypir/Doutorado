\documentclass{beamer}
\usepackage[utf8]{inputenc}
\usepackage{amsmath}
\usepackage{algorithm}
\usepackage{algorithmic}
\usepackage{graphicx}
\usepackage[linesnumbered,ruled,vlined]{algorithm2e}
% Tema da apresentação
\usetheme{CambridgeUS}

% Pacote de cores
\usecolortheme{seahorse}

\title{Robótica Probabilística}
\subtitle{Automação Estocástica com Arduino}
\author{Jefferson Bezerra dos Santos}
\date{}

\begin{document}

\frame{\titlepage}

\section{Introdução}
\begin{frame}{Introdução}
    \begin{enumerate}
        \item \textbf{Objetivo da Pesquisa:} Aperfeiçoar a robótica probabilística utilizando a plataforma Arduino, visando aplicações em contextos técnicos e sociais.
        \item \textbf{Programação Estocástica:} Implementar técnicas de programação estocástica para modelagem e controle de robôs em cenários variados.
        \item \textbf{Desenvolvimento de Protótipos:} Criar e testar protótipos robóticos para avaliar sua adaptação e desempenho em diferentes ambientes.
        \item \textbf{Contribuição Acadêmica:} Produzir publicações e uma tese de doutorado que promovam o estudo da estatística computacional e robótica probabilística.
    \end{enumerate}
\end{frame}

\section{Objetivos}
\begin{frame}{Objetivos}
    \begin{block}{Objetivo Principal}
        Utilizar e aperfeiçoar a robótica probabilística para aplicação em contextos técnicos e sociais variados.
    \end{block}
    \begin{itemize}
        \item Foco na implementação de protótipos com Arduino.
        \item Exploração do uso de programação estocástica na robótica.
    \end{itemize}
\end{frame}


\section{Metodologia}
\begin{frame}{Metodologia}
    \begin{block}{Etapa 1: Previsão}
        Cálculo do estado $x_t$ baseado em $x_{t-1}$ e no controle $u_t$:
        \begin{equation}
            p(x_t \mid x_{t-1}, u_t) = \int p(x_t \mid x_{t-1}, u_t) p(x_{t-1} \mid u_{t-1}) dx_{t-1}
        \end{equation}
    \end{block}
\end{frame}

\begin{frame}{Atualização de Medição}
    \begin{block}{Etapa 2: Atualização}
        Multiplicação pela probabilidade da medição $z_t$ observada:
        \begin{equation}
            p(x_t \mid z_t) \propto p(z_t \mid x_t) p(x_t \mid x_{t-1}, u_t)
        \end{equation}
    \end{block}
\end{frame}

\begin{frame}{Pseudocódigo do Algoritmo}
\begin{algorithm}[H]
\caption{Algoritmo FilterBayes}
    \KwIn{$bel(x_{(t-1)}, u_t, z_t$)}
\KwOut{$bel(x_t)$}
\For{todo $x_t$}{
    \STATE Previsão:  $\overline{bel}(x_t) \gets \int p(x_t \mid u_t, x_{(t-1)}) \, bel(x_{(t-1)}) \, dx_{(t-1)}$\

    \STATE Atualização: $bel(x_t) \gets \eta \cdot p(z_t \mid x_t) \cdot \overline{bel}(x_t)$\
}
\Return{$bel(x_t)$}
\end{algorithm}
\end{frame}

\section{Resultados Esperados}
\begin{frame}{Resultados Esperados}
    \begin{itemize}
        \item Desenvolvimento de protótipos funcionais utilizando Arduino.
        \item Análise da adaptação do robô a diferentes cenários com técnicas de programação estocástica.
        \item Proposição e validação de novos modelos quando necessário.
        \item Investigação detalhada da interação entre o robô e o ambiente.
    \end{itemize}
\end{frame}

\begin{frame}{Publicações e Impacto}
    \begin{itemize}
        \item Elaboração da tese de doutorado com resultados da pesquisa.
        \item Publicações em revistas especializadas e apresentações em congressos.
        \item Contribuição significativa para o estudo em estatística computacional.
    \end{itemize}
\end{frame}

\section{Referências}
\begin{frame}{Referências}
    \footnotesize
    \begin{thebibliography}{99}
        \bibitem{thrun2005probabilistic}
        THRUN, Sebastian; BURGARD, Wolfram; DURRANT-WHYTE, Hugh. \textit{Probabilistic Robotics}. MIT Press, 2005.
        
        \bibitem{cifuentes2019survey}
        CIFUENTES, Mario; SERRANO, Sergio. A survey of probabilistic algorithms for robotics. \textit{Journal of Robotics}, 2019.

        \bibitem{siegmund2019introduction}
        SIEGMUND, Karl; HOFFMANN, Tilo. \textit{Introduction to Robotics: Analysis, Control, Applications}. CRC Press, 2019.
        
        \bibitem{barber2012bayesian}
        BARBER, David. \textit{Bayesian Reasoning and Machine Learning}. Cambridge University Press, 2012.
        
        \bibitem{ross2014}
        ROSS, Sheldon M. \textit{Introduction to Probability Models}. Academic Press, 2014.
    \end{thebibliography}
\end{frame}

\end{document}

