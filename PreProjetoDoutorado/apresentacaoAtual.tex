
\documentclass{beamer}
\usepackage[utf8]{inputenc}
\usepackage{amsmath}
\usepackage{algorithm}
\usepackage{algorithmic}
\usepackage{graphicx}
\usepackage[linesnumbered,ruled,vlined]{algorithm2e}
\usetheme{CambridgeUS}
\usepackage{tikz}
\usetikzlibrary{shapes.geometric, arrows, backgrounds}
\usecolortheme{seahorse}

% Definir paleta de cores
\definecolor{techblue}{RGB}{0, 102, 204}
\definecolor{techgreen}{RGB}{34, 153, 84}
\definecolor{techgray}{RGB}{240, 240, 240}
\definecolor{codebg}{RGB}{30, 30, 30}

% Personalizar estilo dos títulos e subtítulos
\setbeamercolor{frametitle}{bg=techblue, fg=white}
\setbeamercolor{title}{fg=techgreen}
\setbeamerfont{frametitle}{series=\bfseries, family=\sffamily}
\setbeamerfont{title}{series=\bfseries, family=\sffamily}

% Plano de fundo com tema de circuito (opcional)
\setbeamertemplate{background}{
  \begin{tikzpicture}[remember picture, overlay]
    \node[anchor=south west, opacity=0.05] at (current page.south west)
      {\includegraphics[width=\paperwidth,height=\paperheight]{papel.jpg}};
  \end{tikzpicture}
}

% Imagens de cabeçalho
\titlegraphic{\includegraphics[height=1.4cm]{Image/UFRPE.jpg} \hfill
\raisebox{.3\height}{\includegraphics[height=0.8cm]{Image/ppgg.png}}}

\title{Robótica Probabilística}
\subtitle{Automação Estocástica com Arduino}
\author{Jefferson Bezerra dos Santos}
\date{}

\begin{document}

% Slide de título
\frame{\titlepage}

\section*{Sumário}
\begin{frame}{Sumário}
    \tableofcontents
\end{frame}

\section{Introdução}
\begin{frame}{Introdução}
    \begin{minipage}{0.45\textwidth}
        \includegraphics[width=\linewidth]{introducaoImage.jpg}
    \end{minipage}
    \hfill
    \begin{minipage}{0.5\textwidth}
        \begin{quote}
            \normalsize
            \textit{“Robótica probabilística permite que robôs tomem decisões em ambientes incertos...”}
        \end{quote}
        \vspace{0.5cm}
        \hfill \small \textbf{— Sebastian Thrun}
    \end{minipage}
\end{frame}

% Exemplo de bloco com estilo de código
\begin{frame}[fragile]{Exemplo de Código}
    \begin{block}{Código Arduino}
        \begin{verbatim}
        int sensor = 7;
        void setup() {
          pinMode(sensor, INPUT);
        }
        void loop() {
          int leitura = digitalRead(sensor);
          Serial.println(leitura);
        }
        \end{verbatim}
    \end{block}
\end{frame}

% Diagrama de prototipagem com bordas arredondadas
\begin{frame}{Diagrama de Prototipagem}
    \begin{tikzpicture}[node distance=2.0cm, every node/.style={fill=white, font=\scriptsize, text=techblue}, align=center]
        \node (arduino) [rectangle, draw, rounded corners, fill=techgreen!20, minimum width=3cm, minimum height=1cm] {Arduino};
        \node (sensor) [circle, draw, fill=techblue!20, left of=arduino, xshift=-2cm, yshift=1cm] {Sensor\\Ultrassônico};
        \node (motor) [circle, draw, fill=techblue!20, right of=arduino, xshift=2cm, yshift=1cm] {Motor};
        \node (env) [rectangle, draw, rounded corners, fill=techgray!20, below of=arduino, yshift=-1.5cm] {Ambiente};
        \draw[->, thick, techblue] (sensor) -- (arduino) node[midway, above] {Dados\\de Distância};
        \draw[->, thick, techgreen] (arduino) -- (motor) node[midway, above] {Sinal\\de Controle};
        \draw[<->, thick, techblue] (env) -- (sensor) node[midway, left] {Reflexão\\do Som};
        \draw[->, thick, techgreen] (motor) -- (env) node[midway, right] {Movimento};
    \end{tikzpicture}
\end{frame}

% Slide de Conclusão
\begin{frame}{Conclusão e Perguntas}
    \begin{block}{Resumo dos Pontos Principais}
        \begin{itemize}
            \item Introdução à robótica probabilística e à plataforma Arduino.
            \item Implementação prática de algoritmos probabilísticos em protótipos.
            \item Resultados esperados e impactos futuros.
        \end{itemize}
    \end{block}
    \vspace{1cm}
    \centering
    \textbf{\huge Perguntas?}
\end{frame}

\end{document}

