\documentclass[a4paper, 12pt]{article}
\usepackage[utf8]{inputenc}
\usepackage[portuguese]{babel}
\usepackage{amsmath}
\usepackage{graphicx}
\usepackage{listings}
\usepackage{xcolor}
\usepackage{hyperref}

\title{Sistema de Monitoramento Agrícola Inteligente}
\author{Jefferson }
\date{}

\definecolor{codegreen}{rgb}{0,0.6,0}
\definecolor{codegray}{rgb}{0.5,0.5,0.5}
\definecolor{codepurple}{rgb}{0.58,0,0.82}
\definecolor{backcolour}{rgb}{0.95,0.95,0.92}

\lstdefinestyle{mystyle}{
    backgroundcolor=\color{backcolour},   
    commentstyle=\color{codegreen},
    keywordstyle=\color{magenta},
    numberstyle=\tiny\color{codegray},
    stringstyle=\color{codepurple},
    basicstyle=\ttfamily\footnotesize,
    breakatwhitespace=false,         
    breaklines=true,                 
    captionpos=b,                    
    keepspaces=true,                 
    numbers=left,                    
    numbersep=5pt,                  
    showspaces=false,                
    showstringspaces=false,
    showtabs=false,                  
    tabsize=2
}

\lstset{style=mystyle}

\begin{document}

\maketitle

\tableofcontents

\section{Introdução}
Este documento descreve o sistema de monitoramento agrícola implementado para Arduino Nano, com foco nas análises estatísticas utilizadas. O sistema coleta dados de temperatura, umidade do ar e umidade do solo, realizando análises avançadas e armazenando os resultados.

\section{Estatísticas Implementadas}

\subsection{Métricas Descritivas}

\subsubsection{Média Aritmética}
Calculada como:

\[
\bar{x} = \frac{1}{n}\sum_{i=1}^{n}x_i
\]

\begin{lstlisting}[language=C++]
// Cálculo da média no código
float sum = 0;
for (byte i = 0; i < SAMPLE_SIZE; i++) {
    sum += samples[i];
}
res.media = sum / valid_count;
\end{lstlisting}

\subsubsection{Variância e Desvio Padrão}
Variância amostral:

\[
s^2 = \frac{1}{n-1}\sum_{i=1}^{n}(x_i - \bar{x})^2
\]

Desvio padrão:

\[
s = \sqrt{s^2}
\]

\begin{lstlisting}[language=C++]
// Implementação no código
res.variancia = (sum_sq - valid_count * pow(res.media, 2)) / (valid_count - 1);
res.desvio_padrao = sqrt(res.variancia);
\end{lstlisting}

\subsection{Medidas de Posição}

\subsubsection{Quartis}
\begin{itemize}
\item Q1 (Primeiro quartil): 25\% dos dados estão abaixo
\item Mediana (Segundo quartil): 50\% dos dados estão abaixo
\item Q3 (Terceiro quartil): 75\% dos dados estão abaixo
\end{itemize}

\begin{lstlisting}[language=C++]
// Cálculo dos quartis após ordenação
res.q1 = sorted[SAMPLE_SIZE / 4];
res.mediana = sorted[SAMPLE_SIZE / 2];
res.q3 = sorted[3 * SAMPLE_SIZE / 4];
\end{lstlisting}

\subsubsection{Amplitude Interquartil (IQR)}
\[
IQR = Q3 - Q1
\]

\begin{lstlisting}[language=C++]
res.iqr = res.q3 - res.q1;
\end{lstlisting}

\subsection{Identificação de Outliers}
Um valor é considerado outlier se:

\[
x < Q1 - 1.5 \times IQR \quad \text{ou} \quad x > Q3 + 1.5 \times IQR
\]

\begin{lstlisting}[language=C++]
float lower_bound = res.q1 - IQR_FACTOR * res.iqr;
float upper_bound = res.q3 + IQR_FACTOR * res.iqr;
res.is_outlier = (new_val < lower_bound) || (new_val > upper_bound);
\end{lstlisting}

\section{Correlação de Pearson}

\subsection{Fórmula Matemática}
\[
r = \frac{n\sum xy - (\sum x)(\sum y)}{\sqrt{[n\sum x^2 - (\sum x)^2][n\sum y^2 - (\sum y)^2]}}
\]

\subsection{Implementação}
\begin{lstlisting}[language=C++]
float calcularPearson(float x[], float y[], byte n) {
  float sum_x = 0, sum_y = 0, sum_xy = 0;
  float sum_x2 = 0, sum_y2 = 0;
  
  for (byte i = 0; i < n; i++) {
    sum_x += x[i];
    sum_y += y[i];
    sum_xy += x[i] * y[i];
    sum_x2 += x[i] * x[i];
    sum_y2 += y[i] * y[i];
  }
  
  float numerador = n * sum_xy - sum_x * sum_y;
  float denominador = sqrt((n * sum_x2 - sum_x * sum_x) * 
                     (n * sum_y2 - sum_y * sum_y));
  
  return (denominador != 0) ? numerador / denominador : 0;
}
\end{lstlisting}

\subsection{Interpretação}
\begin{itemize}
\item $r \approx 1$: Correlação positiva forte
\item $r \approx -1$: Correlação negativa forte
\item $r \approx 0$: Nenhuma correlação linear
\end{itemize}

\section{Exemplos Práticos}

\subsection{Exemplo de Dados}
\begin{center}
\begin{tabular}{|c|c|c|}
\hline
Temperatura (°C) & Umidade Ar (\%) & Umidade Solo (\%) \\
\hline
25.3 & 62.5 & 45.2 \\
28.4 & 65.2 & 72.3 \\
32.1 & 63.0 & 71.8 \\
\hline
\end{tabular}
\end{center}

\subsection{Cálculo de Estatísticas}
Para a temperatura (25.3, 28.4, 32.1):

\begin{itemize}
\item Média: $(25.3 + 28.4 + 32.1)/3 = 28.6$
\item Variância: $[(25.3-28.6)^2 + (28.4-28.6)^2 + (32.1-28.6)^2]/2 = 11.56$
\item Desvio padrão: $\sqrt{11.56} = 3.4$
\end{itemize}

\subsection{Correlação Calculada}
Entre temperatura e umidade do ar:
\begin{itemize}
\item Coeficiente: -0.72 (correlação negativa moderada)
\item Significativa: SIM (abs(-0.72)$ > $ 0.5)
\end{itemize}

\section{Estrutura do Código}

\subsection{Organização}
\begin{itemize}
\item \textbf{Setup}: Inicialização de sensores e SD card
\item \textbf{Loop principal}:
  \begin{enumerate}
  \item Leitura de sensores
  \item Cálculo de estatísticas
  \item Detecção de outliers
  \item Armazenamento em SD
  \end{enumerate}
\end{itemize}

\subsection{Classes e Estruturas}
\begin{lstlisting}[language=C++]
struct DadosSensores {
  float temperatura;
  float umidade_ar;
  float umidade_solo_percent;
};

struct Estatisticas {
  float media, desvio_padrao, variancia;
  float q1, mediana, q3, iqr;
  bool is_outlier;
};
\end{lstlisting}

\section{Conclusão}
O sistema implementa análises estatísticas robustas para monitoramento agrícola, incluindo:
\begin{itemize}
\item Estatísticas descritivas completas
\item Identificação de valores atípicos
\item Análise de correlação entre variáveis
\end{itemize}

Esta documentação fornece a base teórica e prática para entender e expandir o sistema.

\end{document}
